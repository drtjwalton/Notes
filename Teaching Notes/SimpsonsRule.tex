\documentclass[12pt]{tjwNOTES}

%%% TITLE
\title{Proof of Simpson's Rule}
\author{Dr Timothy J. Walton}
\date{\today}
\pagestyle{headings}

%%% PACKAGE
\usetikzlibrary{intersections, pgfplots.fillbetween}

%%% COMMANDS


%%%%%%%%%%%%%%%%%%%%%%%%%%%%%%%%%%%%%%%%%%%%%%%%%%%%%%%%%%%%%%%%%%%%%%%%
\begin{document}
%%%%%%%%%%%%%%%%%%%%%%%%%%%%%%%%%%%%%%%%%%%%%%%%%%%%%%%%%%%%%%%%%%%%%%%%
\maketitle

We consider the area under the curve $y=f(x)$ where $f$ is a general parabola: 
\begin{align*}
	 f: \RR  &\,\longrightarrow\, \RR \\[0.1cm]
		x 	&\,\longmapsto\,	f(x) \,=\, ax^{2}+bx+c
\end{align*}
for some $a,b,c\in\RR$ with $a\neq 0$. For simplicity, we consider the area under this curve on the interval $[-h,h]$ for some $h>0$:\\[0.3cm]

\begin{center}		
	\begin{tikzpicture}[scale=1]
		\begin{axis}[	height=5cm,
						width=8cm,
						xlabel={\mbox{\large $x$}},
						ylabel={\mbox{\large $y$}},
						xmin=-3, xmax=3,
						ymin=-0.3, ymax=11,
						axis x line = center,
						axis y line = center,
						y axis line style={-{Latex[length=3mm,width=2mm]}, line width=1.1pt},
						x axis line style={-{Latex[length=3mm,width=2mm]}, line width=1.1pt},
						xlabel style={at={(rel axis cs:1.05,-0.05)}, anchor=south},
						ylabel style={at={(rel axis cs:0.49,1.01)}, anchor=south, rotate={0}},
						yticklabels={},
						xtick={-1.5,1.5}, xticklabels={$x=-h$, $x=h$},
						clip=false
						]
			\addplot[red, domain=-2:2, name path=A, line width=1.4pt]{ 8-(x-0.5)^2 };
			%%%
			\draw[black,thick] (axis cs:1.5,0) -- (axis cs:1.5,7);
			\draw[black,fill=white, thick] (axis cs:1.5,7) circle[ radius=2.4pt];
			\draw[black,fill=white, thick] (axis cs:1.5,0) circle[ radius=2.4pt];
			%%%
			\draw[black,thick] (axis cs:-1.5,0) -- (axis cs:-1.5,4);
			\draw[black,fill=white, thick] (axis cs:-1.5,4) circle[ radius=2.4pt];
			\draw[black,fill=white, thick] (axis cs:-1.5,0) circle[ radius=2.4pt];
			%%%
			\draw[black,fill=white, thick] (axis cs:0,7.75) circle[ radius=2.4pt];
			\draw[black,fill=white, thick] (axis cs:0,0) circle[ radius=2.4pt];
			%%%
			\node at (axis cs:2.2,9) {$y=ax^{2}+bx+c$};
			%%%
			\path[name path=axis] (axis cs:-2,0) -- (axis cs:2,0);
			\addplot[color=blue,fill=blue!30!white,opacity=0.4] fill between[of=axis and A, soft clip={domain=-1.5:1.5}];
			%\node at (axis cs:1.7,1.3) {Area, $A(x)$};
		\end{axis}
	\end{tikzpicture}
\end{center}
The curve passes through the points $(-h,y_{0})$, $(0,y_{1})$ and $(h,y_{2})$ where
\begin{align*}
	y_{0} &\,=\, ah^{2} - bh + c	\\
	y_{1} &\,=\, c	\\
	y_{2} &\,=\, ah^{2} + bh + c.
\end{align*}
Considering these as 3 equations in 3 unknowns $a,b,c$, we can solve to find
\begin{align}\label{abc}
	\begin{split}
		a &\,=\, \frac{1}{2h^{2}}(y_{0}-2y_{1}+y_{2}) \\[0.3cm]
		b &\,=\, \frac{1}{2h}(y_{2}-y_{0}) \\[0.3cm]
		c &\,=\, y_{1}.
	\end{split}
\end{align}
The area under the curve $A$ (shaded in the diagram above) is given by the definite integral:
\begin{align*}
	A &\,=\, \int_{x=-h}^{h} (ax^{2}+ bx + c) \,dx \,=\, \left.\frac{1}{3}ax^{3} + \frac{1}{2}bx^{2} + cx\right|_{x=-h}^{h} \\[0.2cm]
		&\,=\, \frac{2}{3}ah^{3} + 2ch \,=\, \frac{h}{3}(2ah^{2} + 6c).
\end{align*}
or after substituting the values of $a,c$ from \eqref{abc}:
\begin{align*}
	A &\,=\, \frac{h}{3}(y_{0}+4y_{1}+y_{2}).
\end{align*}
Now consider the integral of a function $F$: 
\begin{align}\label{INT}
	\int_{x=a}^{b} F(x) \,dx,
\end{align}
where $F$ is assumed to be continuous on $[a,b]$. We divide this interval into $n$ {\bf even} subintervals of equal length:
\begin{align*}
	h &\,=\, \frac{b-a}{n},
\end{align*}
and introduce the $n+1$ points (or ordinates):
\begin{align*}
	x_{0}=a, \; x_{1}=a+h, \; x_{2}=a+2h,\; \ldots,\; x_{n-1}=x+(n-1)h,\; x_{n}=a+nh=b.
\end{align*}
We evaluate the function $F$ at these points:
\begin{align*}
	y_{0}=F(x_{0}), \; y_{1}=F(x_{1}), \; y_{2}=F(x_{2}),\; \ldots,\; y_{n-1}=F(x_{n-1}),\; y_{n}=F(x_{n}).
\end{align*}
\quad\\[-0.2cm]
\begin{center}		
	\begin{tikzpicture}[scale=0.9]
		\begin{axis}[	height=8cm,
						width=18cm,
						xlabel={\mbox{\large $x$}},
						ylabel={\mbox{\large $y$}},
						xmin=0, xmax=22,
						ymin=0, ymax=7,
						axis x line = center,
						axis y line = center,
						y axis line style={-{Latex[length=3mm,width=2mm]}, line width=1.1pt},
						x axis line style={-{Latex[length=3mm,width=2mm]}, line width=1.1pt},
						xlabel style={at={(rel axis cs:1.03,-0.07)}, anchor=south},
						ylabel style={at={(rel axis cs:0,1.0)}, anchor=south, rotate={0}},
						yticklabels={},
						xticklabel style = {font=\scriptsize},
						xtick={4,7,10,13,16}, xticklabels={$x_{1}$, $x_{2}$, $x_{3}$, $x_{4}$, $x_{5}$},
						clip=false
						]
			\addplot[red, domain=0:20, name path=A, line width=1.4pt]{ 0.003*x^3-0.08*x^2+0.55*x+3 };
			\addplot[green, dashed, domain=1:7,line width=3pt]{ -0.04*x^2 + 0.42*x + 3.08 };
			\addplot[violet, dashed, domain=7:13,line width=3pt]{ 0.01*x^2-0.32*x+5.73 };
			\addplot[darkgray, dashed, domain=13:19,line width=3pt]{ 0.064*x^2-1.727*x+14.856 };
			%%%
			\pgfplotsinvokeforeach{1,4,7,10,13,16,19}{%
				\draw[black,thick] (axis cs:#1,0) -- (axis cs:#1,0.003*#1^3-0.08*#1^2+0.55*#1+3);
				\draw[black,fill=white, thick] (axis cs:#1,0.003*#1^3-0.08*#1^2+0.55*#1+3) circle[ radius=3pt];
				\draw[black,fill=white, thick] (axis cs:#1,0) circle[ radius=3pt];
				}
			%%%
			\path[name path=axis] (axis cs:0,0) -- (axis cs:20,0);
			\addplot[color=blue,fill=blue!20!white,opacity=0.4] fill between[of=axis and A, soft clip={domain=1:7}];
			\addplot[color=blue,fill=blue!50!white,opacity=0.4] fill between[of=axis and A, soft clip={domain=7:13}];
			\addplot[color=blue,fill=blue!90!white,opacity=0.4] fill between[of=axis and A, soft clip={domain=13:19}];
			\node at (axis cs:1,-0.45) {\scriptsize $x_{0}=a\phantom{1}$};
			\node at (axis cs:19,-0.45) {\scriptsize $x_{6}=b\phantom{1}$};
			\draw[black,thick, {Latex[length=2mm,width=1.6mm]}-{Latex[length=2mm,width=1.6mm]}] (axis cs:1,3) -- (axis cs:4,3) node[below, pos=0.5]{$h$};
		\end{axis}
	\end{tikzpicture}
	\quad \\[0.4cm]
	\parbox{14cm}{\hspace{2cm}\fbox{\parbox{10cm}{\small	Example of applying Simpson's rule to a function using 6 intervals. The dashed curves denote the quadratic interpolating polynomials between each three successive points. 
											}}}\\[0.2cm]
\end{center}\quad\\[0.2cm]

Since a quadratic polynomial can be used to interpolate three distinct points, we can approximate \eqref{INT} by adding the areas under the parabolic arcs through three successive points:
\begin{align*}
	\int_{x=a}^{b} F(x) \,dx &\,\approx\,  \frac{h}{3}(y_{0}+4y_{1}+y_{2}) +  \frac{h}{3}(y_{2}+4y_{3}+y_{4}) + \cdots +  \frac{h}{3}(y_{n-2}+4y_{n-1}+y_{n}) \\[0.2cm]
		&\,=\, \frac{h}{3}\left(y_{0} + 4y_{1} + 2y_{2} + 4y_{3} + \cdots + 4y_{n-1} + y_{n} \right).
\end{align*}
This is known as {\it Simpson's rule}. 





%%%%%%%%%%%%%%%%%%%%%%%%%%%%%%%%%%%%%%%%%%%%%%%%%%%%%%%%%%%%%%%%%%%%%%%%%%%%%%%%%%%%%%%%%
%%%%%%%%%%%%%%%%%%%%%%%%%%%%%%%%%%%%%%%%%%%%%%%%%%%%%%%%%%%%%%%%%%%%%%%%%%%%%%%%%%%%%%%%%
\end{document}



